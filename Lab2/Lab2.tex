\documentclass{beamer}
\usepackage[russian]{babel}
\usetheme{Boadilla}
\title{Курсовая работа}
\subtitle{Численные методы решения уравнений}
\author{Азаренко Евгений Валерьевич\\КБ1}
\institute{Балтийский федеральный университет имени Иммануила Канта}
\date{\today}

\begin{document}
	
\begin{frame}
	\titlepage
\end{frame}

\begin{frame}
	\frametitle{Описание работы}
	\textbf{Задачи курсовой работы:}
	\begin{itemize}
		\item Цель данной курсовой работы - изучение методов приближённого интегрирования.
		
	\end{itemize}
	
	\textbf{Методы решения:}
	\begin{itemize}
		\item Решение нелинейных уравнений
		\item Метод касательных
		\item Интерполирование функции
		\item Полиномы Ньютона
		\item Численное интегрирование и приближенное решение обыкновенных дифференциальных уравнений первого порядка
		\item Задача Коши
	\end{itemize}
\end{frame}

\begin{frame}
	\frametitle{Результаты работы}
	В ходе выполнения курсовой работы были изучены следующие методы решения профессиональных задач: решение нелинейных уравнений, метод касательных, интерполирование функции, полиномы Ньютона, численное интегрирование и приближенное решение обыкновенных дифференциальных уравнений первого порядка, задача Коши. Было показано, что с помощью данных методов можно достаточно быстро решить многие задачи с указанной степенью точности.
\end{frame}

\end{document}